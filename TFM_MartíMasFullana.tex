\documentclass[a4paper,12pt,twoside]{ThesisStyle}
\usepackage[utf8]{inputenc}
\usepackage{thesis-style}
\usepackage[catalan]{babel}

\begin{document}

\frontmatter

\pagenumbering{gobble}

\thispagestyle{empty}
\begin{table}[htb]
  \centering
  \begin{Large}
    \resizebox{\textwidth}{!}{\begin{tabular}{ | l |}
        \hline
        \\
        \includegraphics[scale=0.9]{imatges/logo_eps.png}    \\[0.7cm]
        \centerline{Treball Final de Màster}                 \\[1cm]
        \hline
        \\
        Estudi: Màster en Ciència de Dades                   \\[0.7cm]
        \hline
        \\
        Títol: Ajustament d'un model generatiu de llenguatge \\
        per a la creació de xatbots personalitzats per       \\
        administracions públiques                            \\[0.7cm]
        \hline
        \\
        Document: Memòria                                    \\[0.7cm]
        \hline
        \\
        Alumne: Martí Mas Fullana                            \\[0.7cm]
        \hline
        \\
        Tutor: Josep Suy Franch                              \\
        Tutor: Miquel Tarragona Margarit                     \\[0.7cm]
        Departament: Departament d'Informàtica, Matemàtica   \\
        Aplicada i Estadística                               \\
        Àrea: Intel·ligència Artificial                      \\[0.7cm]
        \hline
        \\
        Convocatòria (mes/any): Setembre 2024                \\[0.7cm]
        \hline
      \end{tabular}}
  \end{Large}
\end{table}

\newpage
\hypersetup{pageanchor=false}
\begin{titlepage}

  % Upper part of the page
  \includegraphics[scale=0.9]{imatges/logo_eps.png} \\[1cm]
  \begin{center}
    \textsc{\Large Treball Final de Màster} \\[1cm]

    % Title
    \begin{spacing}{2}
      \HRule \\
      \textbf{\Huge Ajustament d'un model generatiu de llenguatge per a la creació de xatbots personalitzats per administracions públiques} \\
      \HRule \\[0.5cm]
    \end{spacing}

    % Author and supervisor and other data
    {
    \large
    \emph{Autor:} \\
    Martí \textsc{Mas Fullana} \\[1cm]
    Setembre 2024 \\[1cm]
    Màster en Ciència de Dades \\[1cm]
    \emph{Tutors:} \\
    Josep \textsc{Suy Franch} \\
    Miquel \textsc{Tarragona Margarit} \\
    }

  \end{center}
\end{titlepage}
\hypersetup{pageanchor=true}

\titlepage

%\dominitoc


\pagenumbering{roman}

\chapter*{Resum}
%\label{cap:resum}



\chapter*{Agraïments}
%\label{cap:agraiments}

Per començar vull agrair molt especialment a \ldots


\tableofcontents

\listoffigures

\listoftables

\mainmatter

\chapter{Introducció}
\label{cap:intro}

\section{Antecedents}
\label{sec:antecedents}

\subsection{Introduction to Dialogue Systems}
\label{subsec:chat}

Dialogue systems, also known as chatbots, have experienced a significant step-change in the last few years. Initially these systems were based on predefined rules and decision trees \cite{Weizenbaum1966ELIZA, AbuShawar2015ALICE}, limiting their capacity for understanding and answering user queries in a natural and flexible manner. These rudimentary systems, commonly referenced as rule-based chatbots, might have been enough for simple tasks, but could not have managed the full complexity and variability of natural language.

\subsection{Towards Language Models}
\label{subsec:language}

As the first machine learning-based language models appeared, such as the Sequence-to-Sequence (Seq2Seq) model \cite{Sutskever2014SequenceSequenceLearningNeural}, and more recently the transformer-based models such as GPT (Generative Pre-trained Transformer) \cite{Vaswani2023AttentionNeed, Radford2018ImprovingLU}, the capacity of chatbots to understand and generate natural language has improved significantly. These models are trained on large datasets of text, learning the complex patterns and structures of language, and are able to generate text that is coherent and contextually relevant.

\subsection{GPT and its Contribution}
\label{subsec:gpt}

The GPT model \cite{Radford2018ImprovingLU}, developed by OpenAI, has been one of the most notable advances in this field. GPT uses the transformer architecture \cite{Vaswani2023AttentionNeed}, which is a type of neural network that is particularly well-suited for processing sequences of data, such as text. Its capacity for generating coherent and contextually relevant responses has been leveraged in a wide range of applications, from virtual assistance to automated content generation.

\subsection{Retrieval Augmented Generation (RAG)}
\label{subsec:rag}

One of the most recent advances in the integration of language models has been the use of retrieval augmented generation (RAG) \cite{Lewis2021RetrievalAugmentedGeneration}. RAG combines the strengths of information retrieval from databases with the generative capacity of language models. In this context, when a user query is received, the system first retrieves relevant information from a database, and then the language model generates a coherent and precise response based on this information. This approach has been shown to improve the accuracy and relevance of the responses generated by chatbots \cite{Lewis2021RetrievalAugmentedGeneration}.

\subsection{Applications and Benefits of RAG-based Chatbots}
\label{subsec:applications}

RAG-based chatbots offer a variety of benefits compared with more traditional systems. They are able to generate responses that are more coherent and contextually relevant. In this way users are both less frustrated and more satisfied. These systems also allow the chatbots to have access to newer, more up to date information than the data the model was originally trained on, as the data provided to the information retrieval component can be updated by simply adding new entries to the database. This makes the chatbot more adaptable and flexible, and allows it to provide more accurate and relevant information to users, reducing the necessity of performing full or partial retraining of the model, which can be prohibitively expensive.

\section{Objectives}

The main goal of this project is to develop an advanced chatbot system that uses GPT (Generative Pre-trained Transformer) technology and RAG (Retrieval Augmented Generation) to provide responses to user queries based off of the content of a database. This general goal can be broken down into the following specific objectives:

\begin{enumerate}
  \item \textbf{Pick an appropriate GPT Model}
        \begin{itemize}
          \item Choose a GPT model that is well-suited for the task of generating responses to user queries based on the content of a database.
        \end{itemize}
  \item \textbf{Integrate RAG Technology}
        \begin{itemize}
          \item \textbf{Information Retrieval:} Develop and implement a system for retrieving relevant information from a database based on user queries.
          \item \textbf{Combine Retrieval and Generation:} Integrate the information retrieval system with the GPT model to generate coherent and contextually relevant responses to user queries.
        \end{itemize}
  \item \textbf{Facilitate User-Chatbot Interaction}
        \begin{itemize}
          \item \textbf{UI Design} Develop a user interface that allows users to interact with the chatbot in a natural and intuitive way.
          \item \textbf{UX Design} Ensure that the user experience is smooth and seamless, and that users are able to easily access the information they need.
        \end{itemize}
  \item \textbf{Accessibility}
        \begin{itemize}
          \item \textbf{Multilingual Support} Implement support for multiple languages to make the chatbot accessible to a wider range of users.
          \item \textbf{Accessibility Features} The system must be designed to be accessible to users with visual or motor impairments. As such, it should support voice input. The voice input feature must be able to be activated through a voice command.
        \end{itemize}
  \item \textbf{Evaluate and Validate the System}
        \begin{itemize}
          \item \textbf{User Testing} Conduct user testing to assess the usability and effectiveness of the chatbot system.
          \item \textbf{Results Analysis} Analyze the results of the different tests to identify areas for improvement and optimization.
        \end{itemize}
\end{enumerate}

\section{Methodology}

\chapter{Estat de l'art}
\label{cap:estat}

\section{Secció}

\subsection{Subsecció}

\chapter{Preliminars}
\label{cap:prelim}

\chapter{Planificació i Metodologia}
\label{cap:plan}

\chapter{Contribució Metodològica}
\label{cap:contrib}

\chapter{Resultats}
\label{cap:result}

\chapter{Conclusions i treball futur}
\label{cap:concl}

\backmatter

%\appendix

%\include{Appendix1}

% \bibliographystyle{ThesisStyleBreakable}
\bibliographystyle{ieeetr}
\bibliography{biblio}

%\printnomenclature

\end{document}
